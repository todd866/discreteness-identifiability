\documentclass[11pt,english,twoside]{article}
\usepackage{babel}
\usepackage{hyperref}
\usepackage{graphicx}
\usepackage{fancyhdr}
\usepackage{amsmath,amssymb}
\usepackage{xcolor}
\usepackage{titlesec}
\usepackage[utf8]{inputenc}
\usepackage{float}
\usepackage{mathtools}
\usepackage[font=scriptsize,labelfont=bf]{caption}
\graphicspath{ {./} {./figures/} }
\titleformat{\section}{\normalsize\bfseries}{\thesection}{1em}{}
\titleformat{\subsection}{\normalsize\bfseries}{\thesubsection}{1em}{}

\usepackage{geometry}
\geometry{
 a4paper,
 total={170mm,257mm},
 left=20mm,
 right=20mm,
 top=15mm,
 bottom=20mm
}

\begin{document}

\thispagestyle{empty}
\setcounter{page}{1}

\pagestyle{fancy}
\fancyhf{}
\fancyhead{}
\fancyhead[RO,LE]{\vspace{15pt}\\Discreteness from Identifiability}
\fancyfoot{}
\fancyfoot[LE,RO]{\thepage}
\fancyfoot[RE,LO]{\url{https://ipipublishing.org/index.php/ipil/}}
\renewcommand{\headrulewidth}{0.4pt}

\begin{minipage}{0.14\textwidth}
\includegraphics[width=0.9\textwidth]{IPI_Pub_Logo.jpg}
\end{minipage}
\hfill
\begin{minipage}{0.5\textwidth}
\includegraphics[width=1.05\textwidth]{IPIL_Logo.jpg}
\end{minipage}
\begin{minipage}{0.3\textwidth}
\begin{flushright}
{\scriptsize
ISSN 2976 - 730X\\
IPI Letters 2026, Vol x (x):x-x\\
\href{https://doi.org/10.59973/ipil.xx}{\color{blue}{https://doi.org/10.59973/ipil.xx}}\\
\medskip
Received: 2026-xx-xx\\
Accepted: 2026-xx-xx\\
Published: 2026-xx-xx\\
}\end{flushright}
\end{minipage}

\vspace{0.5cm}

\par\noindent\rule{\textwidth}{0.5pt}\\
{\color{red}\textbf{Article}}

\begin{center}
\vspace{0.5cm}
{\huge {\bf Discreteness from Identifiability: Why Continuous Physics Produces Integers}}
\vspace{0.5cm}
\end{center}

\noindent
{\large {\bf Ian Todd$^\bold{1,*}$}}

\vspace{0.1in}

\noindent
{\footnotesize $^1$Sydney Medical School, University of Sydney, Sydney, NSW, Australia

\vspace{0.1in}

\noindent
$^*$Corresponding author: \href{mailto:itod2305@uni.sydney.edu.au}{\color{blue}{itod2305@uni.sydney.edu.au}}
}
\vspace{1cm}

\noindent
{\small {\bf Abstract} - Discrete phenomena pervade physics: quantum numbers, particle species, phase transitions, symbolic codes. Yet the underlying state spaces are typically continuous. We identify a unifying mechanism: \emph{discrete transitions emerge from integer-valued invariants of observation maps}. When a continuous high-dimensional system is observed through a constraint, symmetry, or projection, the observation map carries integer-valued invariants---rank, dimension, index---that can change only discretely. The \emph{identifiable dimension} (rank of the observation differential) controls how many effective degrees of freedom are accessible; when this rank changes, qualitative transitions occur. We demonstrate this mechanism in three domains: (1) information geometry, where Fisher rank activation under coupling produces emergent statistical coordinates; (2) quantum mechanics, where measurement partitions state space and integer invariants (degeneracies, quantum numbers) label the structure; (3) symbol formation, where tokens emerge as stable equivalence classes in phase space under finite-resolution observation. We discuss connections to anthropic reasoning: observers require a separation between high-dimensional microstates and lower-dimensional accessible interfaces, suggesting that \emph{effective dimensionality gradients} are necessary for observation. Discreteness is not fundamental but \emph{observational}---a necessary consequence of finite-capacity observers accessing high-dimensional substrates through maps with integer-valued invariants.}

\vspace{0.75cm}

\noindent
{\small {\bf Keywords} - Discreteness; Identifiability; Quotient geometry; Fisher information; Emergence; Observation}

\vspace{0.2cm}
\par\noindent\rule{\textwidth}{0.5pt}

\section{Introduction}

Physics repeatedly produces integers from continuous substrates. Quantum mechanics yields discrete spectra. Statistical mechanics produces phase transitions at critical points. Biological systems generate symbolic codes. Digital computation emerges from analog circuits. In each case, the underlying dynamics are continuous, yet the observed phenomena are discrete.

The standard explanations are domain-specific: quantum discreteness comes from eigenvalue equations; phase transitions from symmetry breaking; codes from evolutionary selection. But the pattern is too universal to be coincidental. We propose a common mechanism: \textbf{discreteness emerges from identifiability}.

The argument is geometric. Consider a smooth map $F: X \to Y$ from a high-dimensional state space $X$ to an observation space $Y$. The differential $dF$ determines which directions in $X$ produce distinguishable outputs in $Y$. The rank of $dF$---the dimension of the identifiable subspace---can only take integer values: $\mathrm{rank}(dF) \in \{0, 1, 2, \ldots, \min(\dim X, \dim Y)\}$.

When parameters or coupling change the structure of $F$, the rank can increase or decrease. But it can only jump by integers. Smooth variation in the underlying physics produces discrete transitions in what can be observed.

This paper develops this insight across three domains and connects it to cosmological structure. The goal is not to replace existing explanations but to identify the common geometric mechanism that makes discrete phenomena \emph{inevitable} whenever continuous systems are observed through finite-capacity channels.

\section{The Quotient-Geometric Mechanism}

\subsection{Observation as Projection}

Let $X$ be a smooth manifold representing the full state space of a physical system, and let $h: X \to Y$ be a smooth observation map. Two states $x_1, x_2 \in X$ are \emph{observationally equivalent} if $h(x_1) = h(x_2)$. This equivalence relation partitions $X$ into fibers:
\begin{align}
    [x] := \{x' \in X : h(x') = h(x)\}.
\end{align}
The quotient space $X/{\sim}$ is the space of distinguishable observations. Under mild regularity conditions, this quotient inherits smooth structure from $X$.

\subsection{Rank as Identifiable Dimension}

The differential $dh_x: T_x X \to T_{h(x)} Y$ determines which infinitesimal variations at $x$ produce distinguishable changes in observation. The \emph{identifiable dimension} at $x$ is:
\begin{align}
    r(x) := \mathrm{rank}(dh_x).
\end{align}
This is the dimension of the subspace of state-space directions that ``matter'' for observation.

\medskip
\noindent\fbox{\parbox{\dimexpr\linewidth-2\fboxsep-2\fboxrule}{%
\textbf{Key property.} The rank function $r: X \to \mathbb{Z}_{\geq 0}$ is integer-valued by definition. It is lower semicontinuous: under limits, rank can \emph{drop} but cannot increase. (The set of points with rank $\le k$ is closed.) Rank deficiencies occur on closed, typically nongeneric sets; changes correspond to structural singularities such as bifurcations, symmetry breaking, or coupling activation.}}
\medskip

\subsection{Why Integer Invariants Matter}

The appearance of integers is not contingent but geometric. Any smooth map between manifolds carries integer-valued invariants: rank, dimension of fibers, dimension of image, indices. These can change only discretely.

\medskip
\noindent\fbox{\parbox{\dimexpr\linewidth-2\fboxsep-2\fboxrule}{%
\textbf{What this does NOT imply.} A quotient structure does not automatically produce a discrete set of equivalence classes. Projecting $\mathbb{R}^2 \to \mathbb{R}$ via $(x,y) \mapsto x$ yields equivalence classes (vertical lines) indexed by $\mathbb{R}$---a continuous quotient. Discreteness of \emph{observed outcomes} requires additional structure: finite resolution, compactness, spectral gaps, or stability constraints. What \emph{is} always discrete are the integer-valued invariants (rank, dimension, index) that characterize the observation map itself.}}
\medskip

The physically significant claim is this: if the identifiable dimension $r(x)$ controls how many effective degrees of freedom are accessible to an observer, then any qualitative change in what can be represented, encoded, or controlled will occur when $r(x)$ changes---and $r(x)$ changes only in integer steps.

\textbf{Proposition (informal).} Let $h: X \to Y$ be an observation map with identifiable dimension $r = \mathrm{rank}(dh)$. Then:
\begin{enumerate}
    \item The number of effective coordinates accessible to the observer is at most $r$.
    \item At finite resolution $\varepsilon$, the number of distinguishable outputs scales as $O(\varepsilon^{-r})$.
    \item Changes in $r$ produce qualitative transitions in the observer's effective description.
\end{enumerate}

This connects integer ranks to observable phenomena: number of order parameters, identifiable couplings, capacity scaling laws, and effective degrees of freedom.

\section{Information Geometry: Rank Activation Under Coupling}

The first application is to coupled dynamical systems, using the information-geometric framework developed in [1].

\subsection{Fisher Rank as Identifiable Dimension}

Consider a family of probability distributions $\{p_\theta : \theta \in \Theta\}$ parameterized by $\theta$. The Fisher information matrix is:
\begin{align}
    g_{ij}(\theta) = \mathbb{E}_{p_\theta}\left[\frac{\partial \log p_\theta}{\partial \theta^i} \frac{\partial \log p_\theta}{\partial \theta^j}\right].
\end{align}
The \emph{Fisher rank} $R(\theta) := \mathrm{rank}(g(\theta))$ counts identifiable parameters---directions in parameter space that produce statistically distinguishable distributions.

\subsection{Coupling Activates Rank}

For uncoupled systems, parameters governing interaction are unidentifiable: they don't affect the observed distribution. Formally, if $\kappa$ is a coupling parameter and the system factors as $p_\theta = p_{\theta_1} \otimes p_{\theta_2}$ at $\kappa = 0$, then:
\begin{align}
    \frac{\partial p_\theta}{\partial \kappa}\bigg|_{\kappa=0} = 0 \implies g_{\kappa\kappa}(0) = 0.
\end{align}
The coupling direction has zero Fisher information; $\kappa$ is not identifiable.

When coupling activates ($\kappa > 0$), the distribution becomes non-factorizable. New directions in parameter space become identifiable. The Fisher rank increases:
\begin{align}
    R(\kappa > 0) > R(0).
\end{align}

This is \textbf{manifold expansion}: coupling reveals new statistical coordinates that were hidden at $\kappa = 0$. The rank jump is discrete---from $R(0)$ to $R(\kappa)$ without intermediate values. A continuous change in coupling produces a discrete change in identifiable dimension.

\subsection{Example: Coupled Oscillators}

In Kuramoto oscillators with coupling $K$, the order parameter $(r, \Psi)$ becomes identifiable only above a critical coupling $K_c$. Below threshold, all coupling values produce statistically indistinguishable outputs (for observations restricted to $(r, \Psi)$). Above threshold, the Fisher rank jumps from 0 to 2. This discrete transition---the synchronization transition---emerges from continuous dynamics through the rank mechanism.

\section{Quantum Mechanics: Integer Invariants from Measurement}

\subsection{Measurement as Quotient}

In quantum mechanics, measurement of observable $\hat{A}$ partitions Hilbert space into eigenspaces:
\begin{align}
    \mathcal{H} = \bigoplus_\lambda \mathcal{H}_\lambda, \quad \mathcal{H}_\lambda := \{|\psi\rangle : \hat{A}|\psi\rangle = \lambda|\psi\rangle\}.
\end{align}
States within the same eigenspace are observationally equivalent with respect to $\hat{A}$: they yield the same measurement outcome with certainty. Measurement induces a quotient structure on state space.

\subsection{When Spectra Are Discrete (and When They Are Not)}

The quotient structure itself does not guarantee discrete outcomes. Many observables have \emph{continuous spectrum}: position, momentum, and the free-particle Hamiltonian all yield continuous ranges of eigenvalues. The quotient labels form a continuum.

Discrete spectra arise from additional structure:
\begin{itemize}
    \item \textbf{Boundary conditions}: confinement to a bounded region quantizes energy levels.
    \item \textbf{Decay conditions}: requiring square-integrability at infinity selects discrete bound states.
    \item \textbf{Compactness}: the hydrogen atom's discrete spectrum reflects the compact effective configuration space.
    \item \textbf{Finite resolution}: even continuous spectra become discrete when binned by finite-precision measurement.
\end{itemize}

What \emph{is} always integer-valued are the invariants characterizing the spectral structure: degeneracies of eigenspaces, dimensions of irreducible representations, quantum numbers labeling symmetry multiplets. These are the integer invariants of quantum mechanics.

\subsection{Collapse as Projection}

Quantum ``collapse'' is projection from the full state to an equivalence class. This is the same geometric move as dimensional reduction: high-dimensional continuous dynamics, viewed through a projection, produce outputs labeled by equivalence classes.

The probabilistic aspect (Born rule) determines \emph{which} class; the quotient structure determines \emph{that} classes exist. Whether the labels form a discrete or continuous set depends on the observable and boundary conditions, not on the quotient structure alone.

\section{Symbol Formation: Tokens as Equivalence Classes}

\subsection{Codes from Dynamics}

Biological and cognitive systems produce discrete symbols (genetic codons, phonemes, concepts) from continuous underlying dynamics. The standard explanation invokes evolutionary selection for error-robustness. But this begs the question: why are discrete codes \emph{possible}?

The answer involves quotient geometry plus finite resolution. A symbol is a stable equivalence class in phase space: a region where many different microstates produce the same functional output under a finite-resolution observation channel.

\subsection{Stability and Finite Resolution}

For a symbol to be reliably transmitted, it must be robust to perturbation. This requires that small changes in the underlying state do not change the symbol---i.e., that the symbol labels an \emph{open} set in state space.

\medskip
\noindent\fbox{\parbox{\dimexpr\linewidth-2\fboxsep-2\fboxrule}{%
\textbf{Required assumptions.} Discrete symbol alphabets require: (1) finite resolution $\varepsilon > 0$ in the observation channel, (2) bounded energy or phase space volume, and (3) stability constraints that collapse continuous microvariation into finitely many robust macrostates.}}
\medskip

Under these assumptions, the number of distinguishable stable classes is finite. Without them, the quotient could be continuous.

\subsection{The Alphabet Size}

Given finite resolution $\varepsilon$ and an observation map $h: X \to Y$ with identifiable dimension $r = \mathrm{rank}(dh)$, the number of distinguishable symbols scales as:
\begin{align}
    |\text{alphabet}| \sim O(\varepsilon^{-r}).
\end{align}
This is a packing-number argument: an $r$-dimensional manifold admits $O(\varepsilon^{-r})$ $\varepsilon$-separated points. Higher rank admits larger alphabets; lower rank forces coarser coding.

This connects to the ``minimal embedding dimension'' results [2]: cyclic processes require $k \geq 3$ for continuous dynamics; below this threshold, dynamics collapse into discrete categories. The discreteness is not chosen but \emph{forced} by dimensional constraints combined with finite resolution.

\section{Discussion: Connections to Anthropic Reasoning}

\textit{This section is more speculative than the preceding technical material. We offer it as a conceptual framework, not a rigorous result.}

\subsection{Observers Require Scale Separation}

The framework developed above suggests a structural requirement for observation: there must be a separation between high-dimensional microstates and lower-dimensional accessible interfaces. An observer is, in this view, a system that compresses high-dimensional environmental states into low-dimensional records (perceptions, measurements, memories).

This is not a claim about literal spacetime dimension. It is a claim about \emph{effective dimensionality}---the rank of the accessible dynamics, the number of identifiable degrees of freedom, the dimension of the coarse-grained description.

\medskip
\noindent\fbox{\parbox{\dimexpr\linewidth-2\fboxsep-2\fboxrule}{%
\textbf{Conjecture (Effective Dimension Gradient).} Observers require an interface where the effective/identifiable dimension of the observation channel is lower than the microstate dimension of the environment, and where this effective dimension can change across conditions (coupling, scale, bifurcation) to support stable memory and control.}}
\medskip

\subsection{Reframing Fine-Tuning}

If observation requires effective-dimension gradients, then fine-tuning questions take a new form:
\begin{itemize}
    \item \textbf{Physical constants} determine whether stable low-dimensional interfaces can form. Some parameter regimes may preclude coarse-grainable structure.
    \item \textbf{Chemistry} is the regime where molecular complexity permits stable symbolic encoding---neither too few degrees of freedom (no capacity) nor too many (no stability).
    \item \textbf{Cosmology} determines where and when effective-dimension gradients arise. Rapid equilibration erases gradients; permanent disequilibrium prevents stable projections.
\end{itemize}

This does not \emph{solve} fine-tuning---it reframes it as a question about the existence of coarse-grainable structure rather than specific parameter values.

\subsection{Why Is There Something? (A Dissolution)}

The question ``why is there something rather than nothing'' may be malformed if ``something'' means ``discrete identifiable structure.'' On our account:
\begin{itemize}
    \item To ask is to observe.
    \item To observe is to project through a finite-capacity channel.
    \item Projection through finite capacity produces discrete outputs (at least: integer-valued invariants).
\end{itemize}

Therefore, any observer will find themselves in a universe with discrete structure. The alternative is not ``nothing''---it is unobservability. This is not an explanation of why the universe exists, but a dissolution of the question: ``nothing'' may not be a coherent observable state.

\subsection{Limitations}

These speculations do not constitute a derivation. We have not shown that effective-dimension gradients are \emph{sufficient} for observers, only that they may be \emph{necessary}. The connection between local differential geometry (rank of maps) and cosmological structure remains to be established rigorously. We flag this section as conceptual scaffolding, not a proven result.

\section{Worked Example: Rank Jump Under Coupling}

To ground the abstract framework, we present a concrete example where rank changes discretely.

\subsection{Setup}

Consider two random variables $X_1, X_2$ with a joint distribution parameterized by $(\mu_1, \mu_2, \kappa)$:
\begin{align}
    X_1 &\sim \mathcal{N}(\mu_1, 1), \quad X_2 \sim \mathcal{N}(\mu_2, 1), \quad \mathrm{Cov}(X_1, X_2) = \kappa.
\end{align}
The coupling parameter $\kappa \in [0,1)$ controls correlation.

\subsection{Observation Map}

Suppose we observe only the sum $Y = X_1 + X_2$. Then:
\begin{align}
    Y \sim \mathcal{N}(\mu_1 + \mu_2, 2 + 2\kappa).
\end{align}
The observation map $h: (\mu_1, \mu_2, \kappa) \mapsto (\mu_1 + \mu_2, 2 + 2\kappa)$ takes 3 parameters to 2 outputs.

\subsection{Rank Analysis}

The Jacobian of $h$ is:
\begin{align}
    dh = \begin{pmatrix} 1 & 1 & 0 \\ 0 & 0 & 2 \end{pmatrix}.
\end{align}
This has $\mathrm{rank}(dh) = 2$ for all $\kappa > 0$: the mean $\mu_1 + \mu_2$ and variance $2 + 2\kappa$ are identifiable, but $\mu_1$ and $\mu_2$ individually are not.

Now suppose at $\kappa = 0$ the system factors: $X_1$ and $X_2$ are independent. If we restrict to the submanifold $\kappa = 0$, the effective parameter space is $(\mu_1, \mu_2)$ and the observation map becomes $h_0: (\mu_1, \mu_2) \mapsto (\mu_1 + \mu_2, 2)$. This has $\mathrm{rank}(dh_0) = 1$: only the sum is identifiable.

\subsection{The Rank Jump}

Crossing from $\kappa = 0$ to $\kappa > 0$:
\begin{itemize}
    \item At $\kappa = 0$: rank = 1 (only sum identifiable)
    \item At $\kappa > 0$: rank = 2 (sum and variance identifiable)
\end{itemize}
This is a discrete transition. Continuous variation in $\kappa$ produces a discrete jump in identifiable dimension. The observer's effective description changes qualitatively: a new coordinate (variance sensitivity) becomes accessible.

This toy example illustrates the general mechanism: coupling activation increases Fisher rank, producing discrete transitions in what observers can learn.

\section{The Structural Spine}

The preceding applications share a common structure:

\begin{center}
\begin{tabular}{|l|l|l|}
\hline
\textbf{Domain} & \textbf{Observation map $h$} & \textbf{Integer invariant} \\
\hline
Information geometry & Score function & Fisher rank \\
Quantum mechanics & Measurement projection & Degeneracy, quantum numbers \\
Symbol formation & Output function & Alphabet size (at finite $\varepsilon$) \\
\hline
\end{tabular}
\end{center}

In each case:
\begin{enumerate}
    \item The substrate is continuous and high-dimensional.
    \item Observation defines an equivalence relation via projection.
    \item The map carries integer-valued invariants (rank, dimension, index).
    \item Structural changes (coupling, measurement, bifurcation) produce discrete transitions in these invariants.
\end{enumerate}

Note: we have removed ``cosmology'' from this table, as the connection between local rank and cosmological structure is conjectural (see Discussion).

This is not a claim that all discrete phenomena reduce to one mechanism. It is a claim that \emph{integer-valued invariants of observation maps} underlie diverse manifestations of discreteness.

\section{Discussion}

\subsection{Discreteness Is Observational, Not Ontological}

We propose a shift in perspective: from ``the world is discrete'' to ``observation produces discreteness.'' The integers in physics are not stamped into reality at the Planck scale; they emerge whenever continuous dynamics are accessed through finite-capacity channels.

This does not make discreteness illusory. The quotient structure is real, objective, and measurable. But it is a property of the observation relation, not of the substrate alone.

\subsection{Relation to the Infodynamics Framework}

This paper extends the infodynamics framework developed in [3,4,5]. The geometric maintenance bound [3] shows that asymmetric states cost work. The aperture theory [4] shows that time is observer-relative information rate. The relaxation framework [5] grounds both in cosmological dynamics.

The present paper adds a complementary insight: the \emph{discreteness} of structure is also observation-dependent. Integers emerge from the rank of the observation map, just as time dilation emerges from aperture contraction.

\subsection{Implications for Quantization}

If discreteness arises from observation rather than substrate, then ``quantization'' may be misnamed. We are not discovering that reality is made of discrete chunks. We are discovering that our observation channels have finite rank, and finite rank produces integer-valued invariants.

This suggests a different approach to quantum gravity: rather than quantizing the metric (making spacetime discrete), we might characterize the observation interface through which the metric becomes accessible [6]. Discreteness would then be a property of the channel, not of spacetime itself.

\section{Conclusion}

We have identified a unifying mechanism for discrete transitions in physics: \textbf{observation maps carry integer-valued invariants that can change only discretely}.

The key invariant is the \emph{identifiable dimension}---the rank of the observation differential---which controls how many effective degrees of freedom are accessible. When this rank changes, qualitative transitions occur in what observers can represent, encode, or control.

This mechanism operates across domains:
\begin{itemize}
    \item In information geometry, Fisher rank counts identifiable parameters; coupling activation produces rank jumps.
    \item In quantum mechanics, degeneracies and quantum numbers are integer invariants of the measurement structure.
    \item In symbol formation, alphabet size at finite resolution scales as $\varepsilon^{-r}$ where $r$ is the identifiable dimension.
\end{itemize}

\medskip
\noindent\fbox{\parbox{\dimexpr\linewidth-2\fboxsep-2\fboxrule}{%
\textbf{What we claim.} Observation maps carry integer-valued invariants (rank, dimension, index). Discrete transitions in observed phenomena correspond to discrete changes in these invariants.\\[0.5em]
\textbf{What we do NOT claim.} Quotients automatically produce discrete equivalence classes. (They don't---see \S2.3.) Continuous spectra don't exist. (They do---see \S4.2.) All fine-tuning reduces to one principle. (It might, but this is conjecture---see \S6.)}}
\medskip

The ubiquity of integers in physics is not mysterious once we recognize that observation maps have integer-valued invariants by geometric necessity. The ``discreteness problem'' shifts from ``why is reality discrete?'' to ``why do observation channels have finite rank?''---and the latter has a straightforward answer: finite-capacity systems cannot resolve infinite-dimensional structure.

\textbf{Discreteness is not fundamental; identifiability is.}

\section*{Acknowledgements}

The author thanks the developers of Claude Code (Anthropic) for assistance with manuscript preparation.

\begin{thebibliography}{6}
{\scriptsize

\bibitem{todd2026manifold}
I. Todd, Manifold expansion via coupling: Fisher rank increases under dynamical interaction, Annals of the Institute of Statistical Mathematics (in preparation, 2026)

\bibitem{todd2025embedding}
I. Todd, Minimal embedding dimension for self-intersection-free recurrent processes, Information Geometry (submitted, 2025)

\bibitem{todd2026infodynamics}
I. Todd, A thermodynamic foundation for the second law of infodynamics, IPI Letters (in press, 2026)

\bibitem{todd2026aperture}
I. Todd, Time as information rate through dimensional apertures: Black hole phenomenology from observer-relative channel capacity, IPI Letters (submitted, 2026)

\bibitem{todd2026cosmic}
I. Todd, Time, mathematics, and the relaxing knot: A geometric foundation for infodynamics, IPI Letters (submitted, 2026)

\bibitem{todd2026gravity}
I. Todd, Gravity as constraint compliance: A thermodynamic framework for quantum gravity, Foundations of Physics (in preparation, 2026)

}
\end{thebibliography}

\end{document}
